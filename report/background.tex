\section{Background}
\label{section:background}

In this section we will describe basic background. First, we will
discuss the problem of mobile Internet and introduce the Host Identity
Protocol. We then move to the discussion of other layer 3 security protocols.
We will conclude the section with the discussion of Elliptic Curves and
a variant of Diffie-Hellman algorithm, which uses EC cryptographgy (ECC).

\subsection{Dual role of IP}

Internet was designed initially so that the Internet Protocol (IP)
address was playing dual role: it was the locator, so that the 
routers could find the recipient of a message, and identifier, so that 
the upper layer protocols (such as TCP and UDP) can make bindings 
(for example, transport layer sockets use IP addresses and ports 
to make a connections). This becomes a problem when a networked 
device roams from one network to another, and so the IP address changes, 
leading to failures in upper layer connections. 

\subsection{Layer 3 security protocols}

There are a lot of solutions today which allow communicating parties 
to authenticate each other and establish secure channel. But only few provide
a separation of identifier and locator.

{\bf Locator Identifier Separation Protocol (LISP)}

{\bf Identifier/Locator Network Protocol (ILNP)}

{\bf Secure Shell protocol (SSH)} is one security solution~\cite{ssh}. SSH is
the application layer protocol which provides an encrypted channel for insecure
networks. SSH was originally designed to provide secure  
remote command-line, login, and command execution. But in fact, 
any network service can be secured with SSH. Moreover, SSH provides
means for creating VPN tunnels between the spatially separated networks.

{\bf IPSec} runs directly on top of the IP protocol and offers two 
various services: (i) it provides the so called Authentication Header (AH),
which is used only for authentication, i.e., it uses various HMAC algorithms, and 
(ii) it provides Encapsulated Security Payload (ESP), which is an authentication 
plus payload encryption mechanism. To establish the security association (negotiate
secret keys and algorithms) one can use IKE or IKEv2, ISAKMP (popular on Windows) or
use preshared keys and set of negotiated algorithms.

{\bf Internet Key Exchange protocol (IKE)}

{\bf Mobile TCP (mTCP)} There are a lot of solutoins for mobility support. 
For sampling see Mobile IP, ROAMIP and Cellular IP.

\subsection{Diffie-Hellman (DH) and Elliptic Curve DH}

Because \texttt{pycryptodome} library does not support 
Diffie-Hellman (DH) and Elliptic Curve Diffie-Hellman (ECDH) algorithms,
we have sat down and derived our own implementation of these
protocols. Here we will mention some background on Elliptic Curve Cryptography (ECC)
and discuss the implementation details of ECDH.

Elliptic curves have the following form $y^2 \equiv x^3+ax+b \mod p$. For the 
curve to have at least one root the discriminant should be non 
zero. In other words, $\Delta = -16(4a^3+27b^2) \not\equiv 0 \mod p$,
where $p$ is a large enough prime number.

By defining a binary operation, which is an addition operation, we can make 
elliptic curve form an abelian group. Remember, abelian group has the following
properties: (i) closure, meaning that if $A, B \in E$, then $A+B\in E$,
(ii) associativity: $\forall A,B,C \in E$ follows that $(A+B)+C=A+(B+C)$.
(iii) existence of identity element $I$, such that $A+I=I+A=A$, (iv)
existence of inverse: $\forall A \in E$ $A+A^{-1}=A^{-1}+A=I$; (v)
commutativity: $A+B=B+A$ $\forall A, B\in E$. Finally, we should 
mention that there should exist an element $G$, such that multiple additions 
of such element with itself, $kG$, generates all other elements of the group. Such groups
are called \texttt{cyclic} abelian groups.

Lets define $O$, a point at infinity, to be identity element, such 
that $P+O = O + P = P, \forall P \in E$. Also, we define 
$P+(-P)=O$, where $-P=(x, -y)$. Next lets 
suppose that $P, Q \in E$ (reads P and Q belong to elliptic curve), 
where $P=(x_1, y_1)$ and $Q=(x_2, y_2)$. We 
can then distinguish the following three cases: (i) $x_1 \neq x_2$ (in this
case the line, which passes through the two given points, 
must intersect the curve somewhere at a third point $R = (x_3, y_3)$), 
(ii) $x_1 = x_2$ and $y_1=-y_2$ (in this case the line is vertical, and it does not
pass through a third point on the curve); and finally (iii) $x_1=x_2$ and $y_1=y_2$
(in this case the line is tangent to a curve, but still crosses the curve at a 
third point $R(x_3, y_3)$). Given case (ii), we can define a negative point 
as $-R = (x, -y)$.

In the first case, line $L$ passes through points $P$ and $Q$. Using simple geometry, 
we can derive an equation of the line as follows: $y = \beta x + \upsilon$, such that
$$\beta = (y_2-y_1)(x_2-x_1)^{-1}$$ Also, we can find $\upsilon$ as 
$$\upsilon=y_1-\beta x_1=y_2-\beta x_2$$

In order to find the points that intersect the curve, we can substitute $y=\beta x+ \upsilon$
into equation of an eliptic curve:

$$(\beta x+ \upsilon)^2=x^3+ax+b$$

By rearranging the terms of the equation, we obtain:

\begin{multline*}
x^3+ax+b-\beta^2x^2-2\beta \upsilon x - \upsilon^2=\\
x^3+(a-2\beta \upsilon)x-\beta^2x^2+b-\upsilon^2=0
\end{multline*}

But since the obtained equation has three roots we have:

\begin{multline*}
(x-x_1)(x-x_2)(x-x_3)=(x^2-xx_2-xx_1+x_1x_2)(x-x_3)=\\
x^3-x^2x_3-x^2x_2+xx_2x_3-x^2x_1+xx_1x_3+xx_1x_2-x_1x_2x_3 = \\
x^3-(x_3+x_2+x_1)x^2+(x_2x_3+x_1x_3+x_1x_2)x-x_1x_2x_3
\end{multline*}

But noticing that the $\beta^2=x_1+x_2+x_3$, we have:

$$x_3=\beta^2-x_1-x_2$$

Moreover, since $P+Q=-R$, we have:

$$-y_3=\beta(x_3-x_1) + y_1$$

or

$$y_3=\beta(x_1-x_3) - y_1$$

The second case is simple, by definition we have $P-Q=O$.
And finally, we should mention that the third case is much like 
the first case, but with one difference - the line that passes through
$P$ and $Q$ is tangent to the curve, because $P=Q$. By applying an
implicit differentiation to an original function of an elliptic curve,
we have:

$$2y\frac{\partial y}{\partial x}=3x^2+a$$

From this we can derive $\beta$ as follows:

$$\beta = \frac{\partial y}{\partial x}=(3x_1^2+a)(2y_1)^{-1}$$.

This expression allows us to derive the $x_3$ as follows:

$$x_3=\beta^2-2x_1$$

Finally, just as in the first case, we have 
$y_3=\beta(x_1-x_3) - y_1$. Of course, all 
operations are done modulo prime $p$.

We now turn to discussion of ECDH protocol and some of the implementation
details. We have used the parameters for the elliptic curve which are 
defined in RFC5903~\cite{RFC5903}. ECDH proceeds in the following manner:
Party $A$ generates random number $i$, where the size of this random number 
is equal to the number of bytes that make up the prime number $p$ (
this is specified in the parameters set). Party $B$ generates in 
a similar fashion number $j$. Both parties, using point on a curve $
G$, which is also a generator (again specified in RFC5903), compute public keys:
$K_A = iG$ and $K_B=jG$. We have used well-known {\bf double and add algorithm}~\cite{stinson} 
to efficiently compute the multiplication. Next parties exchange the public keys and
derive a shared secret as follows $S=iK_B=jK_A=ijG$.

To test the implementation we have used test vectors provided in
previously mentioned RFC.
