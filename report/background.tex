\section{Background}
\label{section:background}

In this section we will describe basic background. First, we will
discuss the problem of mobile Internet and introduce the Host Identity
Protocol. We then move to the discussion of various security protocols.
We will conclude the section with the discussion of Elliptic Curves and
a variant of Diffie-Hellman algorithm, which uses EC cryptography (ECC).

\subsection{Dual role of IP}

Internet was designed initially so that the Internet Protocol (IP)
address is playing dual role: it is the locator, so that the 
routers can find the recipient of a message, and it is an identifier, so that 
the upper layer protocols (such as TCP and UDP) can make bindings 
(for example, transport layer sockets use IP addresses and ports 
to make a connections). This becomes a problem when a networked 
device roams from one network to another, and so the IP address changes, 
leading to failures in upper layer connections. The other problem is
establishment of the authenticated channel between the communicating
parties. In practice, when making connections, long term identities 
of the parties are not verified. Of course, there are solutions such 
as SSL which can readily solve the problem at hand.
However, SSL is suitable only for TCP connections and most of the 
time practical use cases include only secure web surfing and 
establishment of VPN tunnels. Host Identity Protocol on the 
other hand is more flexible: it allows peers to create authenticated secure 
channel on network layer, and so all upper layer protocols can
benefit from such channel.

HIP~\cite{hip} relies on the 4-way handshake to establish
authenticated session. During the handshake, the peers authenticate
each other using long-term public keys and derive session keys 
using Diffie-Hellman or Elliptic Curve (EC) Diffie-Hellman algorithms.
To combat the denial-of-service attacks, HIP also introduces computational 
puzzles.

HIP uses truncated hash of the public key as identifier in a form of IPv6 address 
and exposes this identifier to the upper layer protocols so that applications can make 
regular connections (for example, applications can open regular TCP or UDP socket connections). 
At the same time HIP uses regular IP addresses (both IPv4 and IPv6 are supported) for routing purposes. 
Thus, when the attachment of a host changes (and so does the IP address used 
for routing purposes), the identifier, which is exposed to the applications, stays the same.
HIP uses special signaling routine to notify the corresponding peer about the 
change of locator. More information about HIP can be found in
RFC 7401~\cite{rfc7401}.

\subsection{Secure network protocols}

There are a lot of solutions today which allow communicating parties 
to authenticate each other and establish secure channel. In this section,
we will review some of the most widely used security protocols and discuss
their application use cases. Here, we will also review some of the 
protocols which allow end-hosts to separate the dual role of the IP addresses.

{\bf Secure Shell protocol (SSH)} is one security solution~\cite{ssh}. SSH is
the application layer protocol which provides an encrypted channel for insecure
networks. SSH was originally designed to provide secure  
remote command-line, login, and command execution. But in fact, 
any network service can be secured with SSH. Moreover, SSH provides
means for creating VPN tunnels between the spatially separated networks:
SSH is a great protocol for forwarding local traffic through remote 
server. However, the protocol will fail once the network device changes its 
attachment point in the network. 

{\bf IPSec}~\cite{ipsec} runs directly on top of the IP protocol and offers two 
various services: (i) it provides the so called Authentication Header (AH),
which is used only for authentication, and 
(ii) it provides Encapsulated Security Payload (ESP), which is an authentication 
plus payload encryption mechanism. To establish the security association (negotiate
secret keys and algorithms) one can use IKE or IKEv2, ISAKMP (popular on Windows) or
even preshared keys and set of negotiated beforehand security algorithms. 

{\bf Internet Key Exchange protocol (IKE)}~\cite{rfc5996} is a protocol used in IPSec to establish
security association, just like HIP. Unlike HIP, however, IKE does not solve the 
dual role problem of the IP address.

{\bf Secure socket layer (SSL)} is an application layer solution to secure TCP 
connections. SSL was standardized in RFC 6101~\cite{rfc6101}. And was designed 
to prevent eavesdropping, man-in-the-middle attacks, tampering and message forgery.
In SSL the communicating hosts can authenticate each other with help of 
longer term identities - public key certificates.

Although, the solutions which we have listed above do solve security problems
at various layers of the OSI~\cite{osi} model, they are not designed to deal with node 
mobility. We, therefore, in the paragraphs that follow, list several protocols,
which were designed specifically to support mobility and solve the dual role of 
an IP address.

{\bf Locator Identifier Separation Protocol (LISP)} was specified in RFC 6830~\cite{rfc6830}.
In LISP identifiers and locators can be anything: IPv4 address, IPv6 address, Medium
Access Control address, etc. Some of the advantages, besides separation of locator and
identifier, include: address family traversal (IPv4 over IPv4, IPv4 over IPv6, IPv6 over IPv6,
and even IPv6 over IPv4), mobility, and improved routing scalability.

{\bf Identifier/Locator Network Protocol (ILNP)} was specified in several RFCs. 
It is worth to take a look at RFC 6740~\cite{rfc6740}, which contains
protocol's architectural description. The main advantage of ILNP is that 
it offers incremental deployment, and backwards compatibility with the IP 
protocol. 

{\bf Mobile IP}. Despite that the protocol does not solve
the separation of locator and identifier problem as such, we still
refer it to this category, because it solves issues, which occur
during node mobility. Thus, the protocol was originally designed 
to allow mobile users to move from one network to the other while 
maintaining permanent IP address. The protocol is specified
in RFC 5944~\cite{rfc5944} (for IPv4 protocol) and RFC 6275~\cite{rfc6275} 
(for IPv6 protocol). The main disadvantage of the protocol, however,
is its complexity.

We will now move on to the discussion of the limitations of the cryptography library
which we have used when we were implementing the Host Identity Protocol
using Python language.

\subsection{Diffie-Hellman (DH) and Elliptic Curve DH}

The cryptography library (pycryptodome~\cite{crypto}), which we have used in our 
implementation, at the time of writing, did not support 
Diffie-Hellman (DH) and Elliptic Curve Diffie-Hellman (ECDH) algorithms. Thus, we
we have sat down and derived our own implementation of these
algorithms. To make things a little bit clear, here we will mention 
some background (the reader can look at~\cite{stinson} for more information on 
EC cryptography) on Elliptic Curve Cryptography (ECC)
and discuss the implementation details of ECDH.

Elliptic curves have the following form $y^2 \equiv x^3+ax+b \mod p$. For the 
curve to have at least one root the discriminant should be non 
zero. In other words, $\Delta = -16(4a^3+27b^2) \not\equiv 0 \mod p$,
where $p$ is a large enough prime number.

By defining a binary operation, which is an addition operation, we can make 
elliptic curve form an abelian group. Remember, abelian group has the following
properties: (i) closure, meaning that if $A, B \in E$, then $A+B\in E$,
(ii) associativity: $\forall A,B,C \in E$ follows that $(A+B)+C=A+(B+C)$.
(iii) existence of identity element $I$, such that $A+I=I+A=A$, (iv)
existence of inverse: $\forall A \in E$ $A+A^{-1}=A^{-1}+A=I$; (v)
commutativity: $A+B=B+A$ $\forall A, B\in E$. Finally, we should 
mention that there should exist an element $G$, such that multiple additions 
of such element with itself, $kG$, generates all other elements of the group. Such groups
are called \texttt{cyclic} abelian groups.

Lets define $O$, a point at infinity, to be identity element, such 
that $P+O = O + P = P, \forall P \in E$. Also, we define 
$P+(-P)=O$, where $-P=(x, -y)$. Next lets 
suppose that $P, Q \in E$, where $P=(x_1, y_1)$ and $Q=(x_2, y_2)$. We 
can then distinguish the following three cases: (i) $x_1 \neq x_2$ (in this
case the line, which passes through the two given points, 
must intersect the curve somewhere at a third point $R = (x_3, y_3)$), 
(ii) $x_1 = x_2$ and $y_1=-y_2$ (in this case the line is vertical, and it does not
pass through a third point on the curve); and finally (iii) $x_1=x_2$ and $y_1=y_2$
(in this case the line is tangent to a curve, but still crosses the curve at a 
third point $R(x_3, y_3)$). Given case (ii), we can define a negative point 
as $-R = (x, -y)$.

In the first case, line $L$ passes through points $P$ and $Q$. Using simple geometry, 
we can derive an equation of the line as follows: $y = \beta x + \upsilon$, such that
$$\beta = (y_2-y_1)(x_2-x_1)^{-1}$$ Also, we can find $\upsilon$ as 
$$\upsilon=y_1-\beta x_1=y_2-\beta x_2$$

In order to find the points that intersect the curve, we can substitute $y=\beta x+ \upsilon$
into equation of an elliptic curve:

$$(\beta x+ \upsilon)^2=x^3+ax+b$$

By rearranging the terms of the equation, we obtain:

\begin{multline*}
x^3+ax+b-\beta^2x^2-2\beta \upsilon x - \upsilon^2=\\
x^3+(a-2\beta \upsilon)x-\beta^2x^2+b-\upsilon^2=0
\end{multline*}

But since the obtained equation has three roots, we have:

\begin{multline*}
(x-x_1)(x-x_2)(x-x_3)=(x^2-xx_2-xx_1+x_1x_2)(x-x_3)=\\
x^3-x^2x_3-x^2x_2+xx_2x_3-x^2x_1+xx_1x_3+xx_1x_2-x_1x_2x_3 = \\
x^3-(x_3+x_2+x_1)x^2+(x_2x_3+x_1x_3+x_1x_2)x-x_1x_2x_3
\end{multline*}

By noticing that the $\beta^2=x_1+x_2+x_3$, we have:

$$x_3=\beta^2-x_1-x_2$$

Moreover, since $P+Q=-R$, we have:

$$-y_3=\beta(x_3-x_1) + y_1$$

or

$$y_3=\beta(x_1-x_3) - y_1$$

The second case is simple, by definition we have $P-Q=O$.
And finally, we should mention that the third case is much like 
the first case, but with one difference - the line that passes through
$P$ and $Q$ is tangent to the curve, because $P=Q$. By applying an
implicit differentiation to an original function of an elliptic curve,
we have:

$$2y\frac{\partial y}{\partial x}=3x^2+a$$

From this we can derive $\beta$ as follows:

$$\beta = \frac{\partial y}{\partial x}=(3x_1^2+a)(2y_1)^{-1}$$.

This expression allows us to derive the $x_3$ as follows:

$$x_3=\beta^2-2x_1$$

Finally, just as in the first case, we have 
$y_3=\beta(x_1-x_3) - y_1$. Of course, all 
operations are done modulo prime $p$.

We now turn to discussion of ECDH protocol and some of the implementation
details. We have used the parameters for the elliptic curve which are 
defined in RFC5903~\cite{RFC5903}. ECDH proceeds in the following manner:
Party $A$ generates random number $i$, where the size of this random number 
is equal to the number of bytes that make up the prime number $p$ (
this is specified in the parameters set). Party $B$ generates in 
a similar fashion number $j$. Both parties, using point on a curve $
G$, which is also a generator (again specified in RFC5903), compute public keys:
$K_A = iG$ and $K_B=jG$. We have used well-known {\bf double and add algorithm}~\cite{stinson} 
to efficiently compute the multiplication. Next parties exchange the public keys and
derive a shared secret as follows $S=iK_B=jK_A=ijG$.

To test the implementation we have used test vectors provided in
previously mentioned RFC.
