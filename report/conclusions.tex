\section{Conclusions}
\label{section:conclusion}

In this short article we have discussed the implementation 
details of Host Identity Protocol (HIP), which is a layer $3.5$ 
security solution aiming at separation of dual role of an IP
address. HIP protocol not only solves the problem of separation
of roles of the IP addresses, but can also be a secure mobility
solution. There are many applications of HIP in modern networks -
from establishing a secure channel between stationary network
entities, to securing mobile users.

We have created a minimal Python-based implementation of HIP and 
experimented with it: (i) we have made several microbanchmarkings, and (ii)
we completed several rounds of stress tests of the solution. 
We have also implemented Encapsulated Secure Payload (ESP) for IPSec and 
measured overall performance by performing several rounds of 
bandwidth measurements using \texttt{iperf} tool. 

We hope that the implementation of HIP will be useful and interesting
for other people. For that reason, we have exposed it in public 
repository~\cite{impl}.