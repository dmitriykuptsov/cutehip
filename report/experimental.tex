\section{Experimental evaluation}
\label{section:experiments}

In this section we will discuss the performance related issues of our 
HIPv2 implementation. We begin the discussion with the set of 
microbanchmarkings. Thus, we first evaluate the performance of ECDH and DH
algorithms, then we switch to performance of RSA and ECDSA signature
algorithms, we conclude the discussion with the performance evaluation of AES and HMAC 
algorithms. Afterwards, we present the results for the overall performance 
of the HIP implmentation.

\begin{figure}
	\includegraphics[width=0.5\textwidth]{graphics/puzzle_solution_perf.pdf}
	\caption{Average duration of puzzle solving}
	\label{fig:puzzle}
\end{figure}

To demonstrate the performance of ECHD and DH algorithms we have 
executed the key exchange algorithms $100$ times for various 
groups. Thus, in Figure~\ref{fig:dh} we show the performance of
DH and in Figure~\ref{fig:ecdh} we show the performance 
of ECDH for various curve parameters. To understand how two
are related in Table~\ref{tab:strength} we show the sizes
of various keys and how they are related to symmetric keys.
Obviously, ECDH shows far better performance than regular
DH algorithm. This performance improvement is largely due
to reduced key sizes.

\begin{figure}
	\includegraphics[width=0.5\textwidth]{graphics/dh_computation_hist.pdf}
	\caption{Diffie-Hellman key exchange duration (total)}
	\label{fig:dh}
\end{figure}

\begin{figure}
	\includegraphics[width=0.5\textwidth]{graphics/ecdh_computation_hist.pdf}
	\caption{Elliptic Curve Diffie-Hellman key exchange duration (total)}
	\label{fig:ecdh}
\end{figure}

\begin{table}
\centering
\begin{tabular}{|c|c|c|}
\hline
\bf{Symmetric key sizes, bits} & \bf{DH keys, bits} & \bf{ECDH keys, bits} \\\hline
		80			&    1024                        & 160                                  \\
		112			&    2048                        & 224                                  \\
		128			&    3072                        & 256                                  \\
		192			&    7680                        & 384                                  \\
		256			&    15360                       & 521                                  \\
\hline
\end{tabular}
\caption{Security strength of keys}
\label{tab:strength}
\end{table}

Next, we have measured the computational performance of other 
cryptographic primitives. Thus, in Table~\ref{tab:micro} we 
show summary statistics for all the operations we have completed.
We have performed $100$ rounds of measurements for each 
cryptographic primitive. The size of the data for all operations
was selected to be $1500$ bytes.

\begin{table*}
\centering
\begin{tabular}{|c|c|c|c|c|}
\hline
\bf{Operation} & \bf{Mean time, ms} & \bf{Median time, ms} & \bf{Standard deviation, ms}\\\hline\hline
RSA (2048 bits, SHA-256) signing					&    $2.099$                        & $2.094$               & $0.026$                  \\
RSA (2048 bits, SHA-256) verification			&    $0.591$                        & $0.589$               & $0.008$                  \\
ECDSA (secp384r1, SHA-384) signing		&    $1.379$                        & $1.374$               & $0.042$                  \\
ECDSA (secp384r1, SHA-384) verification	&    $3.008$                        & $3.006$               & $0.016$                  \\
AES-256 encryption  &    $0.036$                  & $0.031$         & $0.027$           \\
AES-256 descryption &	 $0.032$ 				  & $0.029$         & $0.017$           \\
HMAC (SHA-256)      &    $0.057$                  & $0.054$         & $0.018$           \\
\hline
\end{tabular}
\caption{Performance of cryptographic primitives}
\label{tab:micro}
\end{table*}

\begin{figure}
	\includegraphics[width=0.5\textwidth]{graphics/packet_processing.pdf}
	\caption{Packets' processing time}
	\label{fig:packet_processing}
\end{figure}

We have ran the HIPv2 BEX for 20 times and measured the total packet processing time (we have combined packet 
processing time for initiator and responder). In Figure~\ref{fig:packet_processing} we show the boxplots for 
the packet processing duration. To run the tests we have used the following configuration: for signatures 
we have used RSA with $2048$ bits long modulus, SHA-256 for HMAC and hashing, ECDH with NIST521 curve, 
AES-256 for encryption and $16$ bits for puzzle difficulty. We have noticed that processing R1 packet consumes considerable
amount of time on responder. Since our implementation was lacking pre-creation of R1 packets, such lengthy packet 
processing time was expected. We have also measured the overall duration of the HIPv2 base exchange (BEX). 
In Figure~\ref{fig:duration_bex} we demonstrate distribution of HIP BEX durations. Clearly, implementing 
cryptographic protocols in userspace using high-level languages, such as Python, is not the best choice: the performance
of such implementations is somewhat unacceptable for production servers and it is better to implement the security 
solutions for overloaded servers using lower level languages, such as C or C++. On the otherside, Python implmentation
is more suitable for study and experimental purposes.

\begin{figure}
	\includegraphics[width=0.5\textwidth]{graphics/duration_bex.pdf}
	\caption{Duration of HIPv2 BEX}
	\label{fig:duration_bex}
\end{figure}

Finally, we have measured throughput for TCP connections over IPSec tunnel and plain TCP. We have used
\texttt{iperf}~\cite{iperf} tool to measure throughput. Clearly, with our implementation we were able to achieve
throughput slightly $25$ times less, than throughput, which we have obtained for plain TCP connections.
Such result was expected for Python implementations of security protocols. For example, we have 
observed similar behaviour with our Python implmentation of VPN solution in one of our previous 
studies~\cite{vpn}.

\begin{figure}
	\includegraphics[width=0.5\textwidth]{graphics/throughput.pdf}
	\caption{Obtained throughput for TCP over IPSec and plain TCP connections}
	\label{fig:throughput}
\end{figure}
