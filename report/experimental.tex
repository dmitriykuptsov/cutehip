\section{Experimental evaluation}
\label{section:experiments}

In this section we will discuss the performance related issues of our 
HIPv2 implementation. We begin the discussion with the set of 
microbanchmarkings. Thus, we first evaluate the perormance of ECDH and DH
algorithms, then we switch to performance of RSA, DSA, and ECDSA signature
algorithms, we conclude the discussion with the performance of various HMAC 
algorithms.

\begin{figure}
	\includegraphics[width=0.5\textwidth]{graphics/puzzle_solution_perf.pdf}
	\caption{Average duration of puzzle solving}
	\label{fig:puzzle}
\end{figure}

To demonstrate the performance of ECHD and DH algorithms we have 
executed the key exchange algorithms $100$ times for various 
groups. Thus, in Figure~\ref{fig:dh} we show the performance of
DH and in Figure~\ref{fig:ecdh} we show the performance 
of ECDH for various curve parameters. To understand how two
are related in Table~\ref{tab:strength} we show the sizes
of various keys and how they are related to symmetric keys.
Obviously, ECDH shows far better performance than regular
DH algorithm. This performance improvement is largely due
to reduced key sizes.

\begin{figure}
	\includegraphics[width=0.5\textwidth]{graphics/dh_computation_hist.pdf}
	\caption{Diffie-Hellman key exchange duration (total)}
	\label{fig:dh}
\end{figure}

\begin{figure}
	\includegraphics[width=0.5\textwidth]{graphics/ecdh_computation_hist.pdf}
	\caption{Elliptic Curve Diffie-Hellman key exchange duration (total)}
	\label{fig:ecdh}
\end{figure}

\begin{table}
\centering
\begin{tabular}{|c|c|c|}
\hline
\bf{Symmetric key sizes, bits} & \bf{DH keys, bits} & \bf{ECDH keys, bits} \\\hline
		80			&    1024                        & 160                                  \\
		112			&    2048                        & 224                                  \\
		128			&    3072                        & 256                                  \\
		192			&    7680                        & 384                                  \\
		256			&    15360                       & 521                                  \\
\hline
\end{tabular}
\caption{Security strength of keys}
\label{tab:strength}
\end{table}

