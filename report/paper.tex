\documentclass[conference,10pt,letter]{IEEEtran}

\usepackage{url}
\usepackage{amssymb,amsthm}
\usepackage{graphicx,color}

%\usepackage{balance}

\usepackage{cite}
\usepackage{amsmath}
\usepackage{amssymb}

\usepackage{color, colortbl}
\usepackage{times}
\usepackage{caption}
\usepackage{rotating}
\usepackage{subcaption}

\newtheorem{theorem}{Theorem}
\newtheorem{example}{Example}
\newtheorem{definition}{Definition}
\newtheorem{lemma}{Lemma}

\newcommand{\XXXnote}[1]{{\bf\color{red} XXX: #1}}
\newcommand{\YYYnote}[1]{{\bf\color{red} YYY: #1}}
\newcommand*{\etal}{{\it et al.}}

\newcommand{\eat}[1]{}
\newcommand{\bi}{\begin{itemize}}
\newcommand{\ei}{\end{itemize}}
\newcommand{\im}{\item}
\newcommand{\eg}{{\it e.g.}\xspace}
\newcommand{\ie}{{\it i.e.}\xspace}
\newcommand{\etc}{{\it etc.}\xspace}

\def\P{\mathop{\mathsf{P}}}
\def\E{\mathop{\mathsf{E}}}

\begin{document}
\sloppy
\title{Implementing Host Identity Protocol using Python}
\maketitle
\begin{abstract}
Host Identity Protocol, or HIP, is layer 3.5 solution,
which was initially designed to split the dual role of
the IP address - locator and identifier. Using HIP protocol
one can solve not only mobility problems, but also 
establish authenticated secure channel. In this short 
report we will introduce description of the implementation
of HIP and IPSec protocols using Python. We will also present the 
microbanchmarking results for various cryptographic primitives,
and present the results for overall performance of HIP and IPSec,
which we implement in userspace using Python language.
\end{abstract}
\input intro.tex
\input background.tex
\input hardware.tex
\input experimental.tex
\input conclusions.tex
%\balance
\bibliographystyle{abbrv}
\bibliography{mybib}

\end{document}

