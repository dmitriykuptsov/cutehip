\documentclass[conference,10pt,letter]{IEEEtran}

\usepackage{url}
\usepackage{amssymb,amsthm}
\usepackage{graphicx,color}

%\usepackage{balance}

\usepackage{cite}
\usepackage{amsmath}
\usepackage{amssymb}

\usepackage{color, colortbl}
\usepackage{times}
\usepackage{caption}
\usepackage{rotating}
\usepackage{subcaption}
\usepackage{flushend}

\newtheorem{theorem}{Theorem}
\newtheorem{example}{Example}
\newtheorem{definition}{Definition}
\newtheorem{lemma}{Lemma}

\newcommand{\XXXnote}[1]{{\bf\color{red} XXX: #1}}
\newcommand{\YYYnote}[1]{{\bf\color{red} YYY: #1}}
\newcommand*{\etal}{{\it et al.}}

\newcommand{\eat}[1]{}
\newcommand{\bi}{\begin{itemize}}
\newcommand{\ei}{\end{itemize}}
\newcommand{\im}{\item}
\newcommand{\eg}{{\it e.g.}\xspace}
\newcommand{\ie}{{\it i.e.}\xspace}
\newcommand{\etc}{{\it etc.}\xspace}

\def\P{\mathop{\mathsf{P}}}
\def\E{\mathop{\mathsf{E}}}

\begin{document}
\sloppy
%\balance
\title{Experimenting with Python implementation of Host Identity Protocol}
\author{Dmitriy Kuptsov}
\maketitle
\begin{abstract}
Host Identity Protocol, or HIP, is a layer 3.5 solution and 
was initially designed to split the dual role of the IP 
address - locator and identifier. Using HIP protocol one 
can solve not only mobility problems but also establish 
authenticated secure channel between two communicating 
end-hosts. In this short article, we first introduce relevant 
background information. We then present some mathematical 
background related to Elliptic Curve (EC) Cryptography and 
Diffie-Hellman protocol based on EC. Finally, we demonstrate 
some micro benchmarking results for various cryptographic 
primitives and conclude the paper with the results for the 
overall performance of HIP and IPSec, which we implemented 
in Linux userspace using Python language.
\end{abstract}
\input intro.tex
\input background.tex
\input hardware.tex
\input experimental.tex
\input conclusions.tex
%\balance
\bibliographystyle{abbrv}
\bibliography{mybib}

\end{document}

